\documentclass[12pt, article]{scrartcl}
\usepackage[english]{babel}
\usepackage{sectsty}
\allsectionsfont{\centering \normalfont\scshape}
\usepackage{fancyhdr}
\pagestyle{fancyplain}
\fancyhead{}
\fancyfoot[L]{}
\fancyfoot[C]{}
\fancyfoot[R]{\thepage}
\renewcommand{\headrulewidth}{0pt}
\renewcommand{\footrulewidth}{0pt}
\setlength{\headheight}{13.6pt}
\newcommand{\horrule}[1]{\rule{\linewidth}{#1}}

\title{	
\normalfont \normalsize 
\textsc{University of Idaho: CS 470 - Artificial Intelligence} \\ [25pt]
\horrule{0.5pt} \\[0.4cm]
\huge Project 4: Prolog\\
\horrule{2pt} \\[0.5cm]
}
\author{Andrew Schwartzmeyer}
\date{\normalsize\today}

\begin{document}
\maketitle 
\begin{abstract}
For this project I wrote a Prolog knowledge base, that given a set of parent and gender facts (that is, a fact for each parental relationship, and a gender for each person), family-relationship queries can be answered. Defined relationships range from as specifc as brothers to as general as ancestors. Specifically, this knowledge base contains most of my family within four generations.
\end{abstract}
\pagebreak
\section{Introduction}
The knowledge base consists of two fact types: parental relationships and genders. A parent fact is expressed as ``parent(a, b)'' where ``b'' is the child of ``a'' (these cannot be capitalized, as in Prolog, words starting with a capitalized letter are considered variables, which are used in rules). There are two gender facts, male and female. A gender is expressed as ``male(a) or female(a)'', where ``a'' is one person.

The rules in the knowledge base are defined abstractly based on facts, other rules, or combinations of the two. For instance, a sibling relationship exists when there is a common parent to two persons, and a brother relationship exists where X is a sibling to Y, and X is also male. A more complex relationship would be an uncle. An uncle relationship between X and Y is exists when there is a brother relationship between X and some Z, and Y is a child of said Z.

Possibly the most complexly defined relationship is that of second cousins, where X and Y share a greatgrandparent (which was defined off of the grandparent relationship, off the parent fact), and are also not siblings, cousins, nor the same person.

Ancestory is a recursively defined relationship, with a base case of an ancestor being a parent X to Y, and then defined on top of itself to include parents of ancestors. This made defining descendants easy, as it is simply the opposite of the ancestor relationship.

The most general relationship is that of a relative, that is, X is a relative to Y (and vice versa) if they share some Z ancestor.

Some of the relationships defined are mutual. If there is some sibling, cousin, or relative relationship between two people in X and Y, then the relationship also exists from Y to X, and will be returned seemingly twice by the Prolog interpreter; this is expected behavior.

\section{Relationships}
The defined relationshps are: mother, father, child, gender specifc child (son or daughter), partner (that is, share a child), grandparent, gender specific and maternal/paternal versions of grandparent, gender specific versions of grandchild, great-great-grandparents and children, ancestor, descendant, relative, sibling, gender specific sibling (sister or brother), uncle, aunt, cousin, cousin once-removed (children of cousins), and second cousins (sharing only a great-grandparent).

\section{Results}
\end{document}
